\documentclass[12pt]{report}
\usepackage{url}
\setlength{\oddsidemargin}{0.25in}
\setlength{\textwidth}{6in}
\setlength{\topmargin}{-0.5in}
\setlength{\textheight}{8.5in}
\newcommand{\Lower}[1]{\smash{\lower 1.5ex \hbox{#1}}}
\newenvironment{reflist}{\begin{list}{}{\listparindent -.5in
\leftmargin 0.5in}
\item \ \vspace{-.35in} }{\end{list}}
\newcommand{\E}{\mbox{E}}
\newcommand{\Var}{\mbox{Var}}
\newcommand{\Cov}{\mbox{Cov}}
\newcommand{\Vec}[1]{\bm{#1}}
\usepackage{enumitem}
\usepackage{bm}
\usepackage[utf8]{inputenc}

\usepackage{listings}
\usepackage{xcolor}

\definecolor{codegreen}{rgb}{0,0.6,0}
\definecolor{codegray}{rgb}{0.5,0.5,0.5}
\definecolor{codepurple}{rgb}{0.58,0,0.82}
\definecolor{backcolour}{rgb}{0.95,0.95,0.92}
\lstdefinestyle{mystyle}{
    backgroundcolor=\color{backcolour},   
    commentstyle=\color{codegreen},
    keywordstyle=\color{magenta},
    numberstyle=\tiny\color{codegray},
    stringstyle=\color{codepurple},
    basicstyle=\ttfamily\footnotesize,
    breakatwhitespace=false,         
    breaklines=true,                 
    captionpos=b,                    
    keepspaces=true,                 
    numbers=left,                    
    numbersep=5pt,                  
    showspaces=false,                
    showstringspaces=false,
    showtabs=false,                  
    tabsize=2
}

\lstset{style=mystyle}
% Set the beginning of a LaTeX document
\begin{document}

\centerline{ {\Large
\textbf{Homework \#1}
}}
\centerline{ {\Large 
\textbf{MATH 7360 -- Fall 2020}
}}
\centerline{ {
\textbf{Due: Friday, Sep 11, 2020}
}}

\paragraph{Some R exercises}

\begin{enumerate}
\item Let $a = 0.7$, $b = 0.2$,  and $c = 0.1$. 
     \begin{enumerate}[label=(\alph*)]
     	\item Write out $0.7$, $0.2$, and $0.1$ in binary.
    	\item In R, test whether $(a + b) + c$ equals 1.
	\item In R, test whether $a + (b + c)$ equals 1.
	\item In R, test whether $(a + c) + b$ equals 1.
	\item Explain what you found.  Hint: find out how addition is performed on numerics (double).
     \end{enumerate}

\item Create the vector $\Vec{x} = (0.988, 0.989, 0.990, \dots, 1.010, 1.011, 1.012)$.
    \begin{enumerate}[label=(\alph*)]
	\item Plot the polynomial $y = x^7 - 7 x^6 + 21 x^5 - 35x^4 + 35x^3 - 21 x^2 + 7x -1$ at points $x_i$ in $\Vec{x}$.
	\item Plot the polynomial $y = (x - 1)^7$ at points $x_i$ in $\Vec{x}$.
	\item Explain what you found.
    \end{enumerate}

\item Let $\Vec{u} = (1, 2, 3, 3, 2, 1)^\top$.  
    \begin{enumerate}[label={\alph*}]
        \item Compute $\Vec{U} = \Vec{I} - (2/d) \Vec{u} \Vec{u}^\top$ where $d = \Vec{u}^\top \Vec{u}$.  (This type of matrix is known as an `elementary reflector' or a `Householder transformation'.)
        \item Let $\Vec{C} = \Vec{U}  \Vec{U}$, the matrix product of $\Vec{U}$ and itself.  Find the largest and smallest off-diagonal elements of $\Vec{C}$.
        \item Find the largest and smallest diagonal elements of $\Vec{C}$.
        \item Compute $\Vec{U} \Vec{u}$.  (matrix times vector).
        \item Compute the scalar $\max_i \sum_j |U(i, j)| $.
        \item Print the third row of $\Vec{U}$.
        \item Print the elements of the second column below the diagonal.
        \item Let $\Vec{A}$ be the first three columns of $\Vec{U}$.  Compute $\Vec{P} = \Vec{A} \Vec{A}^\top$.
        \item Show that $\Vec{P}$ is idempotent (in other words $\Vec{P} = \Vec{P}\Vec{P}$) by recomputing (e) with $\Vec{P}\Vec{P} - \Vec{P}$.
        \item Let $\Vec{B}$ be the last three columns of $\Vec{U}$.  Compute $\Vec{Q} = \Vec{B}\Vec{B}^\top$.
        \item Show that $\Vec{Q}$ is idempotent by recomputing (e) with $\Vec{Q} \Vec{Q} - \Vec{Q}$.
        \item Compute $\Vec{P} + \Vec{Q}$.
    \end{enumerate}
    
\item Read in the matrix in the file `oringp.dat' on the failure of O-rings leading to the Challenger disaster.
The columns are flight number, date, number of O-rings, number failed, and temperature at launch.
Compute the correlation between number of failures and temperature at launch, deleting the last, missing observation (the disaster).

\item Functions
    \begin{enumerate}[label={\alph*}]
        \item What are the three components of a function?
        \item What does the following code return?

\begin{lstlisting}[language=R]
x <- 10
f1 <- function(x) {
    function() {
        x + 10
    }
}
f1(1)()
\end{lstlisting}
	
	\item How could you make this call easier to read?
\begin{lstlisting}[language=R]
mean(, TRUE, x = c(1:10, NA))
\end{lstlisting}
	
	\item Does the following function throw an error when called?  Why/why not?

\begin{lstlisting}[language=R]
f2 <- function(a, b) {
  return(a * 10)
}
f2(10, stop("This is an error!"))
\end{lstlisting}
        
    \end{enumerate}

\item Let the $n \times n$ matrix $\Vec{A}$ have elements $A(i, j) = 1/(|i - j| + 1)$.
    \begin{enumerate}[label = {\alph*}]
        \item Create a function that takes input argument $n$ and output matrix $\Vec{A}$.
        \item Compute and print $\Vec{A}$ for $n = 10$.
        \item Compute and print the Cholesky factorization for $\Vec{A}$ for $n = 10$.  Hint: try chol() function.
        \item Find the determinant of $\Vec{A}$.
    \end{enumerate}
 \end{enumerate}

\end{document}